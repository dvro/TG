% !TEX encoding = ISO-8859-1
\chapter{T�cnicas de Sele��o de Prot�tipos}
% \label{ch:introducao}

Neste cap�tulo, ser�o mostradas as t�cnicas de sele��o de prot�tipos abordadas neste trabalho. Cada uma das sess�es abaixo abordar� uma t�cnica, ser� mostrado o conceito da t�cnica, assim como o pseudo-c�digo e as caracter�sicas de cada uma destas t�cnicas.

\section{ENN}

	Edited Nearest Neighbor Rule\cite{enn:2011} � uma t�cnica de sele��o de prot�tipos puramente seletiva proposta por Wilson em 1976. De uma forma geral, esta t�cnica foi projetada para funcionar como um filtro de ru�dos, ela elimina pontos na regi�o de fronteira, regi�o de alta susceptibilidade a erros, e com isso elimina ru�dos.
	Por atuar apenas na regi�o de fronteira, esta t�cnica possui uma baixa capacidade de redu��o, deixando as inst�ncias que n�o se encontram na regi�o de fronteira intactas, exceto pelos ru�dos extremos.
	Uma desvantagem desta t�cnica � que ela possui uma baixa capacidade de redu��o de elementos, visto que ela n�o elimina redund�ncia.

	Segue abaixo o algor�tmo da execu��o do ENN e, logo ap�s, alguns coment�rios sobre este algor�tmo.



\section{Tomek Links}
\section{CNN}
\section{LVQ}
\subsection{LVQ 1}
\subsection{LVQ 2.1}
\subsection{LVQ 3}
\section{SGP}
\section{SGP 2}
\section{CCNN}


