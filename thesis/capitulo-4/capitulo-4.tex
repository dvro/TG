% !TEX encoding = ISO-8859-1
\chapter{Bases Reais}
\label{ch:experimentobasesreais}

Neste cap�tulo ser�o mostrados experimentos feitos com bases reais. As bases utilizadas foram algumas das utilizadas por Fern�ndez em seu estudo \textit{Sistemas de classifica��o baseados em fuzzy hier�rquico com regra de sele��o gen�tica para bases desbalanceadas}\cite{FJH2009}.

Estas bases reais foram obtidas como parte do software KEEL, que cont�m um m�dulo de experimentos em bases desbalanceadas. Para todos os experimentos foi utilizado \textit{K-Fold Cross-Validation}, com 5 folds, respeitando a propor��o das classes em todas as divis�es, formato j� fornecido junto com o software.


\section{Iris 0}

A base Iris0 � uma base com baixo desbalanceamento. Esta base cont�m 150 inst�ncias, cada uma contendo quatro atributos, sendo 33.33\% destas inst�ncias da classe minorit�ria, \textit{Iris-Setosa}, e 66.66\% da classe marjorit�ria, \textit{reminder}.


\begin{table}[H]\tiny
\begin{center}
\begin{tabular}{|@{}l@{}|@{}l@{}|@{}l@{}|@{}l@{}|@{}l@{}|@{}l@{}|@{}l@{}|@{}l@{}|@{}l@{}|@{}l@{}|@{}l@{}|}
\hline
T�cnica		   				&	KNN		&	ENN		&		CNN		&	Tomek Links		&	OSS	&	LVQ 1	&	LVQ 2.1	&	LVQ 3	&	SGP	&	SGP 2 	\\
\hline %----- linha horizontal
Acerto Total				& 100.00 $\pm$ 0.00 \%	& 100.00 $\pm$ 0.00 \%& 67.39 $\pm$ 2.31 \%& 100.00 $\pm$ 0.00 \%& 84.18 $\pm$ 27.88 \%& 99.31 $\pm$ 1.54 \%& 99.31 $\pm$ 1.54 \%& 99.31 $\pm$ 1.54 \%& 98.64 $\pm$ 1.86 \%& 98.64 $\pm$ 1.86 \% \\
\hline
Acerto Marjorit�ria			& 100.00 $\pm$ 0.00 \%& 100.00 $\pm$ 0.00 \%& 100.00 $\pm$ 0.00 \%& 100.00 $\pm$ 0.00 \%& 76.00 $\pm$ 42.63 \%& 98.95 $\pm$ 2.35 \%& 98.95 $\pm$ 2.35 \%& 98.95 $\pm$ 2.35 \%& 97.95 $\pm$ 2.81 \%& 97.95 $\pm$ 2.81 \% \\
\hline
Acerto Minorit�ria		    & 100.00 $\pm$ 0.00 \%& 100.00 $\pm$ 0.00 \%& 0.00 $\pm$ 0.00 \%& 100.00 $\pm$ 0.00 \%& 100.00 $\pm$ 0.00 \%& 100.00 $\pm$ 0.00 \%& 100.00 $\pm$ 0.00 \%& 100.00 $\pm$ 0.00 \%& 100.00 $\pm$ 0.00 \%& 100.00 $\pm$ 0.00 \% \\
\hline
Tamanho Resultante			& 100.00 $\pm$ 0.00 \%& 100.00 $\pm$ 0.00 \%& 2.55 $\pm$ 0.02 \%& 100.00 $\pm$ 0.00 \%& 34.18 $\pm$ 0.79 \%& 8.50 $\pm$ 0.06 \%& 8.50 $\pm$ 0.06 \%& 8.50 $\pm$ 0.06 \%& 1.70 $\pm$ 0.01 \%& 1.70 $\pm$ 0.01 \% \\
\hline
\end{tabular}%--- fechaoento do aobiente tabular
\end{center}   %fio da centraliza��o da tabela
\caption{Tabela do iris0}
\label{tab:iris}
\end{table}



\section{Glass 5}

A base Glass5 � uma base com alto desbalanceamento. Esta base cont�m 214 inst�ncias, cada uma contendo nove atributos. Apenas 4.20\% das inst�ncias s�o da classe minorit�ria, \textit{tableware}, e 95.80\% da classe marjorit�ria, \textit{reminder}.


\begin{table}[H]\tiny
\begin{center}
\begin{tabular}{|@{}l@{}|@{}l@{}|@{}l@{}|@{}l@{}|@{}l@{}|@{}l@{}|@{}l@{}|@{}l@{}|@{}l@{}|@{}l@{}|@{}l@{}|}
\hline
T�cnica		    &KNN\%&ENN\%&CNN\%&Tomek Links\%&OSS\%&LVQ 1\%&LVQ 2.1\%&LVQ 3\%&SGP\%&SGP 2\% \\
\hline %----- linha horizontal
TOTAL		    & 96.28 $\pm$ 3.53 \%& 95.81 $\pm$ 2.55 \%& 95.80 $\pm$ 1.02 \%& 96.28 $\pm$ 3.53 \%& 85.53 $\pm$ 15.62 \%& 80.35 $\pm$ 4.65 \%& 81.28 $\pm$ 5.81 \%& 81.30 $\pm$ 5.98 \%& 95.80 $\pm$ 1.93 \%& 95.80 $\pm$ 1.93 \% \\
\hline
Acerto Marjorit�ria		   & 98.05 $\pm$ 3.18 \%& 99.02 $\pm$ 2.18 \%& 100.00 $\pm$ 0.00 \%& 98.05 $\pm$ 3.18 \%& 86.83 $\pm$ 16.60 \%& 80.98 $\pm$ 5.56 \%& 81.95 $\pm$ 6.36 \%& 81.46 $\pm$ 6.59 \%& 99.51 $\pm$ 1.09 \%& 99.51 $\pm$ 1.09 \% \\
	\hline
Acerto Minorit�ria		    & 60.00 $\pm$ 41.83 \%& 30.00 $\pm$ 44.72 \%& 0.00 $\pm$ 0.00 \%& 60.00 $\pm$ 41.83 \%& 60.00 $\pm$ 41.83 \%& 60.00 $\pm$ 54.77 \%& 60.00 $\pm$ 54.77 \%& 70.00 $\pm$ 44.72 \%& 10.00 $\pm$ 22.36 \%& 10.00 $\pm$ 22.36 \% \\
\hline
Tamanho Resultante		     & 100.00 $\pm$ 0.00 \%& 95.78 $\pm$ 1.13 \%& 7.86 $\pm$ 1.24 \%& 99.30 $\pm$ 0.64 \%& 10.67 $\pm$ 0.88 \%& 5.86 $\pm$ 0.03 \%& 5.86 $\pm$ 0.03 \%& 5.86 $\pm$ 0.03 \%& 3.17 $\pm$ 1.59 \%& 1.64 $\pm$ 1.46 \% \\
\hline
\end{tabular}%--- fechaoento do aobiente tabular
\end{center}   %fio da centraliza��o da tabela
\caption{Tabela do glass5}
\end{table}



\section{Yeast 6}

A base Yesat 6 � a base de mais alto n�vel de desbalanceamento estudada neste trabalho. Esta base cont�m 1484 inst�ncias, cada uma contendo 8 atributos. Apenas 2.49\% das inst�ncias s�o da classe minorit�ria, \textit{exc}, e 97.51\% da classe marjorit�ria, \textit{reminder}.




